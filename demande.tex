\documentclass[11pt]{article}
\usepackage{fontspec,xunicode,polyglossia,longtable,csquotes}
\setmainfont{Linux Libertine O}
\setmainlanguage{french}
\usepackage[a4paper]{geometry}
\begin{document}
\title{Journées \enquote{\LaTeX\ et sciences humaines et sociales}}
\author{Responsables : Joël Gombin (CURAPP - UPJV) \\ et Maïeul Rouquette (IRSB - FTRS - Université de Lausanne)}
\maketitle


Cet événement se déroulera durant deux jours à l'automne 2013, à l'Université de Picardie Jules Verne (Amiens). 

\vspace{12pt}


Objectifs :
\begin{itemize}
\item Formation à \LaTeX\ (niveau débutant et avancé).
\item Réflexion sur les interactions avec d'autres logiciels.
\item Réflexion sur la place des logiciels libres et de standards ouverts dans la recherche et l'enseignement en SHS.
\item Mise en place d'une Association des utilisateurs de \LaTeX\ en SHS.
\end{itemize}

\section{Programme prévisionnel}

Les journées alterneront séance en plénière et ateliers pratiques.
\begin{longtable}{lc|c}
Horaire                               & Atelier 1             &   Atelier 2 \\
\hline
\endhead
 & \multicolumn{2}{c}{\textbf{Journée 1}}  \\
9h-9h30                                     &  \multicolumn{2}{c}{Petit déjeuner }     \\          
9h30-10h                                   & \multicolumn{2}{c}{Accueil}               \\ 
10h-11                                        &   \multicolumn{2}{c}{Panorama des usages de \LaTeX\ en SHS (1) }  \\
11h-11h15                                &   \multicolumn{2}{c}{Pause} \\                
11h15-12h15                            &   \multicolumn{2}{c}{Panorama des usages de \LaTeX\ en SHS (2) + revues.org}  \\               %%% pq revues.org ? Très bonne question...
12h15-14h30                            &   \multicolumn{2}{c}{Repas}                \\
14h30-16h                                & Les Bases de LaTeX        &      Créer des classes \LaTeX \\
16h-16h30                         &   \multicolumn{2}{c}{Pause}\\             
16h30-18h30                  &   \multicolumn{2}{c}{Assemblée générale de fondation de l'association}\\             
18h30-19h30        &  \multicolumn{2}{c}{Visite de la ville} \\                
19h30-23h00        & \multicolumn{2}{c}{Repas festif} \\
                        
 & \multicolumn{2}{c}{\textbf{Journée 2}}  \\                
                        
9h-9h30                        & \multicolumn{2}{c}{Petit déjeuner }  \\             
9h30-10h30                & \multicolumn{2}{c}{Bibliographie (1)}  \\  
10h30-10h45            &   \multicolumn{2}{c}{Pause} \\     
10h45-12h15            & {Bibliographie (2)} & Git et github     \\        
12h-14h                    & \multicolumn{2}{c}{Repas} \\               
14h30-16h00        & édition critique en \LaTeX\  &        \LaTeX\ et R\\
16h-16h30        &\multicolumn{2}{c}{Pause}            \\
16h30-18h00    &    \LaTeX\ et XML TEI       & {Export vers d'autres formats} \\

\end{longtable}
\section{Budget}

\subsection{Dépenses}

\begin{longtable}{lcr}
Poste & Détail & Total \\
\hline
\endhead
Transport des intervenants     & 10 x 200 €     & 2000 € \\
Logement des intervenants     & 10 x 90 €        & 900 € \\ %(3 nuits, si doivent arriver la veiller et repartir le lendemain)
Sandwichs/buffet pour le midi & 40 x 2 x 5 €    & 160 € \\
Repas festif                                 & 40  x 15 €      & 600 € \\
Café, gâteaux                               & 40 x 2 x 5 €   & 160 € \\
\hline
\textbf{Total}                                &                         & 3850 € \\
\end{longtable}

\subsection{Ressources}

\begin{longtable}{lr}
Poste & Total \\
\hline
\endhead

CURAPP & 1850 € \\
Conseil scientifique & 1000 € \\
École doctorale & 1000 € \\
\hline
\textbf{Total} & \textbf{3850 €} \\

\end{longtable}


\section{Besoins matériels}
\begin{itemize}
\item 2 salles : 1 pour 40 p. et 1 pour 20 p.
\item 2  vidéos-projecteurs
\item Diffusion
\item Un endroit où servir les gâteaux et le café (bibliothèque du CURAPP par ex.) 
\end{itemize}
\end{document}